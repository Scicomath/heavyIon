\documentclass{ctexart}
\usepackage{amsmath}
\usepackage{amssymb}
\usepackage{bm}
\usepackage{geometry}

\geometry{a4paper,left=3cm,right=3cm}

\title{相对论重离子碰撞作业}
\author{艾鑫}

\newcounter{mycnt}
\setcounter{mycnt}{0}
\newenvironment{problem}{\noindent \stepcounter{mycnt}\themycnt.}{

}
\newenvironment{answer}{\textbf{解}:}{
\vspace{0.5cm}
}

\newcommand\diff{\,\mathrm{d}}

\begin{document}
\maketitle

\begin{problem}
  证明$x_+, x_-$是洛伦兹不变量, 证明快度是洛伦兹变换可加量.
\end{problem}

\begin{answer}
  考虑参考系$F$和$F'$, $F'$系相对于$F$系以速度$\beta$沿着$x$轴匀速运动. 在$F$系中的四矢量$c = (c_0, \bm{c}_T, c_z)$, 在$F'$系中变为$c' = (c_0', \bm{c}_T', c_z')$. 它们两者由洛伦兹变换联系起来:
  \begin{gather}\label{c0}
    c_0' = \gamma (c_0 - \beta c_z), \\ \label{c1}
    c_z' = \gamma (c_z - \beta c_0), \\
    \bm{c}_T' = \bm{c}_T.
  \end{gather}
其中
\begin{equation}
  \gamma = \frac{1}{\sqrt{1-\beta^2}}.
\end{equation}

将\eqref{c0}和\eqref{c1}相加得到:
\begin{equation}
  c_0' + c_z' = \gamma (1 - \beta) (c_0 + c_z). \label{c0cz}
\end{equation}
因此$c_0' + c_z'$与$c_0 + c_z$直相差一个因子$\gamma (1 - \beta)$. 类似的, 另外一个粒子$b$在$F$系的前向光锥动量$b_0 + b_z$与在$F'$系中前向光锥动量$b_0' + b_z'$的关系为:
\begin{equation}
  b_0' + b_z' = \gamma (1 - \beta) (b_0 + b_z). \label{b0bz}
\end{equation}
由$x_+$的定义有
\begin{gather}
  x_+ = \frac{c_0 + c_z}{b_0 + b_z} \\
  x_+' = \frac{c_0' + c_z'}{b_0' + b_z'}
\end{gather}
由\eqref{c0cz}和\eqref{b0bz}很容易得到:
\begin{equation}
  x_+ = x_+'
\end{equation}
因此, 光锥变量$x_+$是洛伦兹不变量. 同理可证, $x_-$也是洛伦兹不变量.

下面证明快度是洛伦兹可加量. 首先在$F$系中的快度有:
\begin{equation}
  y = \frac{1}{2} \ln \left( \frac{p_0 + p_z}{p_0 - p_z}\right).
\end{equation}
在$F'$系中有:
\begin{equation}
  y' = \frac{1}{2} \ln \left( \frac{p_0' + p_z'}{p_0' - p_z'}\right).
\end{equation}
由洛伦兹变换有:
\begin{gather}
  p_0' = \gamma (p_0 - \beta p_z) \\
  p_z' = \gamma (p_z - \beta p_0).
\end{gather}
因此
\begin{equation}
  \begin{split}
    y' &= \frac{1}{2} \ln \left[ \frac{\gamma(1-\beta)(p_0 + p_z)}{\gamma(1+\beta)(p_0 - p_z)} \right] \\
    &= y + \frac{1}{2} \ln \left( \frac{1-\beta}{1+\beta} \right) \\
    &= y - \frac{1}{2} \ln \left( \frac{1+\beta}{1-\beta} \right)
  \end{split}
\end{equation}
由上式可知, $y'$与$y$只相差一个因子$\frac{1}{2}\ln((1+\beta)/(1-\beta))$, 因此快度是洛伦兹不变量.

\end{answer}

\begin{problem}
  在无限大动量参照系中, 讨论相对论硬散射过程, $A + B \rightarrow C + X$. 证明:
  \begin{enumerate}
  \item[(1)] $b^2 = [x_b(1-x_b)B^2 - x_b \beta^2 - b_T^2]/(1 - x_b)$
  \item[(2)] $a^2 = [x_a(1-x_a)A^2 - x_a \alpha^2 - a_T^2]/(1 - x_a)$
  \end{enumerate}
\end{problem}

\begin{answer}
  考虑入射粒子$B$和靶核$A$, 在无限大动量参考系中$A$和$B$的动量可写为:
  \begin{gather} \label{eqB}
    B = (B_0, \bm{B}_T, B_z) = (P_1 + \frac{B^2+B_T^2}{4P_1}, \bm{B}_T, P_1 - \frac{B^2+B_T^2}{4P_1}) \\ \label{eqA}
    A = (A_0, \bm{A}_T, A_z) = (P_2 + \frac{A^2+A_T^2}{4P_2}, \bm{A}_T, -P_2 + \frac{A^2+A_T^2}{4P_2}) \\
  \end{gather}

对于反应$A + B \rightarrow C + X$, 我们可以认为$C$是由$A$的部分子$a$和$B$的部分子$b$作用形成的$c$, 然后进一步反应形成的. 
我们引入$b$相对于$B$的前向光锥动量比$x_b$:
\begin{equation} \label{eqxb}
  x_b = \frac{b_0 + b_z}{B_0 + B_z}, 
\end{equation}
同样, 我们定义$a$相对于$A$的后向光锥动量比$x_a$:
\begin{equation} \label{eqxa}
  x_a = \frac{a_0 - a_z}{A_0 - A_z}.
\end{equation}
利用公式\eqref{eqB}和\eqref{eqxb}, 我们可以写出$b$的动量:
\begin{equation}
  b = \left(  x_bP_1 + \frac{b^2+b_T^2}{4x_bP_1}, \bm{b}_T, x_bP_1 - \frac{b^2+b_T^2}{4x_bP_1} \right). 
\end{equation}
类似的, 我们可以写出$a$的动量:
\begin{equation}
  a = \left(  x_aP_2 + \frac{a^2+a_T^2}{4x_aP_2}, \bm{a}_T, -x_aP_2 + \frac{a^2+a_T^2}{4x_aP_2} \right). 
\end{equation}

因此动量$\beta = (B - b)$和$\alpha = (A - a)$可以写为:
\begin{gather} \label{eqbeta}
  \beta = \left(  (1-x_b)P_1 + \frac{\beta^2+\beta_T^2}{4(1-x_b)P_1}, -\bm{b}_T, (1-x_b)P_1 - \frac{\beta^2+\beta_T^2}{4(1-x_b)P_1} \right),  \\ \label{eqalpha}
  \alpha = \left(  (1-x_a)P_2 + \frac{\alpha^2+\alpha_T^2}{4(1-x_a)P_2}, -\bm{a}_T, -(1-x_a)P_2 + \frac{\alpha^2+\alpha_T^2}{4(1-x_a)P_2} \right).  \\
\end{gather}

由于$b = B - \beta$, 我们可以得到:
\begin{equation}
  b^2 = B^2 - 2B\cdot \beta + \beta^2.
\end{equation}
利用公式\eqref{eqB}和\eqref{eqbeta}我们可以得到:
\begin{equation}
  \begin{split}
    B\cdot\beta &= \left(P_1 + \frac{B^2}{4P_1})((1-x_b)P_1 + \frac{\beta^2+\beta_T^2}{4(1-x_b)P_1}\right) - \left(P_1 - \frac{B^2}{4P_1})((1-x_b)P_1 - \frac{\beta^2+\beta_T^2}{4(1-x_b)P_1}\right) \\
    &= \left[(1-x_b)P_1^2 + \frac{\beta^2+\beta_T^2}{4(1-x_b)} + \frac{(1-x_b)B^2}{4} + \frac{B^2(\beta^2+\beta_T^2)}{16(1-x_b)P_1^2}\right] - \\
    & \quad \left[(1-x_b)P_1^2 - \frac{\beta^2+\beta_T^2}{4(1-x_b)} - \frac{(1-x_b)B^2}{4} + \frac{B^2(\beta^2+\beta_T^2)}{16(1-x_b)P_1^2}\right] \\
    &= \frac{\beta^2 + \beta_T^2}{2(1-x_b)} + \frac{(1-x_b)B^2}{2} \\
    &= \frac{\beta^2 + \beta_T^2 + (1-x_b)^2B^2}{2(1-x_b)}.
  \end{split}
\end{equation}
因此,
\begin{equation}
  \begin{split}
    b^2 &= B^2 + \beta^2 - 2B\cdot \beta \\
    &= B^2 + \beta^2 - \frac{\beta^2 + \beta_T^2 + (1-x_b)^2B^2}{1-x_b} \\
    &= \frac{B^2(1-x_b) + \beta^2(1-x_b) - [\beta^2 + \beta_T^2 + (1-x_b)^2B^2]}{1-x_b} \\
    &= \frac{x_b(1-x_b)B^2 - x_b \beta^2 -\beta_T^2}{1-x_b} \\
    &= \frac{x_b(1-x_b)B^2 - x_b \beta^2 - b_T^2}{1-x_b}.
  \end{split}
\end{equation}
同理可证, 对$a$有
\begin{equation}
  a^2 = \frac{x_a(1-x_a)A^2 - x_a \alpha^2 - a_T^2}{1-x_a}.
\end{equation}
\end{answer}

\begin{problem}
  在QED理论中, 有一个费米场$\psi$和电磁场$A^\mu$, $A^\mu$又称为规范场, 试证明无质量费米子的电磁相互作用, 等同于一个质量$m = e/\sqrt{\pi}$的自由玻色场$\phi$, $e$为电磁耦合常数.
\end{problem}

\begin{answer}
  我们从充满电子的狄拉克负能海出发, 如果在某个空间区间内有一个电荷密度或电流的扰动, 将会产生一个电磁规范场$A^\mu$, 它将影响费米子场算符, 结果费米子场的变化将使所有的电子处于各种运动和激发状态. 显然, 这种运动和激发会产生电荷密度$j^0$和电流$j^1$. 下面我们来求产生的电流$j^\mu$和扰动源---规范场$A^\mu$的依赖关系.

首先,我们注意到这两个量在规范变换下是不一样的. 在量子点动力学中, 流$j^\mu$这样的物理量是规范不变量, 即它们在规范场从$A^\mu$到$\bar{A}^\mu$按如下规范变换时保持不变,
\begin{equation}
  \bar{A}^\mu(x) = A^\mu (x) - \partial^\mu \lambda(x),
\end{equation}
而费米子场算符$\psi(A)$的变换为,
\begin{equation}
  \psi(x,\bar{A}(x)) = e^{ie\lambda(x)} \psi(x, A(x)),
\end{equation}
其中$\lambda(x)$是一个$x$的任意函数. 对任意函数$\lambda(x)$的不同选择代表不同的规范, 而物理量必须与规范的选择无关.

尽管产生的流$j^\mu(x)$依赖于它的规范场源$A^\mu(x)$, 但是$j^\mu(x)$和$A^\mu(x)$在规范变换下的行为又是不同的, 流是规范不变的, 而规范场却依赖于规范的选择. 当考虑了流的规范不变性后, 产生的流$j^\mu(x)$与局域电磁扰动$A^\mu(x)$的关系为
\begin{equation}\label{jmu}
  j^\mu(x) = - \frac{e^2}{\pi} [A^\mu(x) - \partial^\mu \frac{1}{\partial^\lambda\partial_\lambda} \partial_\nu A^\nu(x)].
\end{equation}

由规范场$A^\mu$产生的流$j^\mu$反过来又是规范场$A^\mu$的源, $j^\mu$确定的规范场$A^\mu$满足麦克斯韦方程:
\begin{equation}
  \partial_\nu F^{\mu\nu} = \partial_\nu (\partial^\mu A^\nu - \partial^\nu A^\mu) = -j^\mu.
\end{equation}

当由$j^\mu$产生的规范场$A^\mu$和由\eqref{jmu}式引入的电磁场$A^\mu$是自洽一致的时候, 我们可以得到这个系统的动力学. 利用这个条件, 我们得到一个描述规范场$A^\mu$动力学的运动方程:
\begin{equation}
  \partial_\nu \partial_\mu A^\nu - \partial_\nu \partial^\nu A^\mu = \frac{e^2}{\pi}[A^\mu - \partial^\mu \frac{1}{\partial^\lambda \partial_\lambda} \partial_\nu A^\nu],
\end{equation}
该方程成立的条件是
\begin{equation}
  -\square A^\mu - \frac{e^2}{\pi}A^\mu = 0.
\end{equation}
这里, 算符$\square$代表$\partial_\nu \partial^\nu$, 它在坐标表象中等于算符$-p^2$.

用$p^2$来表示, 上式可以被写做:
\begin{equation}
  p^2 A^\mu - \frac{e^2}{\pi} A^\mu = 0.
\end{equation}
我们可以将上式与Klein-Gordon方程比较, 如果$A^\mu$是一个质量为$m$的自由玻色子场, 有
\begin{equation}
  p^2A^\mu - m^2 A^\mu = 0.
\end{equation}
可以得到, 规范场$A^\mu$满足Klein-Gordon方程就像是一个自由的玻色子, 其具有质量
\begin{equation}
  m = \frac{e}{\sqrt{\pi}}.
\end{equation}
因此, 包含无质量费米子的$\mathrm{QED}_2$等效于一个具有质量$e/\sqrt{\pi}$的自由玻色子场.


\end{answer}

\begin{problem}
  利用$1+1$维相对论理想流体力学方程, 导出QGP物质熵随时间的变化关系, 以及发生QGP到强子物质相变, 发生相变时间.
\end{problem}

\begin{answer}
  我们用Bjorken的流体力学模型来概括在演化的流体动力学相中的等离子体动力学, 其中等离子体被理想地描述为一种相对论气体. 在初始$\tau = \tau_0$时, 等离子体达到局域热平衡, 且初始能量密度为$\epsilon_0$, 初始温度$T(\tau_0)$正比于$\epsilon_0^{1/4}$. 随后, 能量密度及压力随固有时以$\tau^{-4/3}$下降,而温度以$\tau^{-1/3}$下降. 

下面我们来求熵密度与固有时的关系. 在不变的温度和压力下, 能量的变化与体积和熵的变化关系为
\begin{equation}
  \diff E = -P \diff V + T \diff S.
\end{equation}
这样可以写出用$\epsilon$和$P$表示熵密度$s = \diff S / \diff V$的关系式为
\begin{equation}
  s \equiv \frac{\diff S}{\diff V} = \frac{\epsilon + P}{T}.
\end{equation}
因而, 熵密度与固有时的关系为
\begin{equation}
  \frac{s(\tau)}{s(\tau_0)} = \frac{\epsilon(\tau) + P(\tau)}{\epsilon(\tau_0) + P(\tau_0)} \frac{T(\tau_0)}{T(\tau)} = \left( \frac{\tau_0}{\tau} \right)^{4/3} \left( \frac{\tau}{\tau_0} \right)^{1/3} = \frac{\tau_0}{\tau}.
\end{equation}
因此, 熵密度反比与固有时.

下面我们来求从QGP物质到强子物质的相变时间. 随着QGP的演化, 它的温度按$\tau^{-1/3}$下降, 当等离子体的温度降到$T_c$时, 固有时$\tau_c$为
\begin{equation}
  \tau_c = \left( \frac{T(\tau_0)}{T_c} \right)^3 \tau_0.
\end{equation}
随后将发生从QGP到强子物质的相变.

\end{answer}

\begin{problem}
  试证明考虑screening效应后, $c$与$\bar{c}$在QGP环境中, 对应的Yukawa势为$V(r) = \frac{q}{4\pi} \frac{e^{-r/\lambda_D}}{r}$.
\end{problem}

\begin{answer}
  我们考虑在$\bm{r} = 0$处的粲夸克$c$. 由于强相互作用, 粲夸克的存在会吸引等离子体中的反夸克, 并排斥夸克. 因此使得周围的介质极化. 下面我们将证明, 对于无质量夸克和反夸克气体的等离子体的理想情况, 在$\bm{r}$探测夸克所感受到的势$V(\bm{r})$将由库伦势被修正为Yukawa势.

在阿贝尔近似下, 在$\bm{r}$的色屏蔽势$V(\bm{r})$来自于以下三种贡献: (1) 在原点$\bm{r} = 0$的$c$产生的势$V_0(\bm{r})$, (2) 等离子体中的夸克产生的势$V_q(\bm{r})$, (3) 等离子体中的反夸克产生的势$V_{\bar{q}}(\bm{r})$,因此
\begin{equation}\label{eq1}
  V(\bm{r}) = V_0(\bm{r}) + V_q(\bm{r}) + V_{\bar{q}}(\bm{r}).
\end{equation}
这样, 在$\bm{r}$的一个夸克将受到一个力$-q\nabla V(\bm{r})$, 而在$\bm{r}$的一个反夸克将受到一个力$-(-q)\nabla V(\bm{r})$.

我们考虑在$\bm{r}$的一个流元, 它含有密度为$n_q$的夸克和密度为$n_{\bar{q}}$的反夸克, 并受到在$\bm{r}$的势$V(\bm{r})$的力. 因此这个流元的每单位体积将受到作用于其中夸克上的力$-qn_q \nabla V(\bm{r})$和作用于其中反夸克上的力$-(-q)n_{\bar{q}} \nabla V(\bm{r})$. 这个流元还将受到由于存在夸克$c$而引起的夸克和反夸克空间再分布的每单位体积的力$\nabla P$.

在$\bm{r}$的流元处于平衡态的条件是作用在其上(单位体积)的合力为零,
\begin{equation}\label{eq2}
  \nabla P(n_q(\mu),n_{\bar{q}}(\mu)) - qn_q(\mu) \nabla V - (-q)n_{\bar{q}}(\mu)\nabla V = 0.
\end{equation}
为了简单, 我们考虑夸克是极端相对论情况, 这时夸克和反夸克的静止质量可以忽略, 压力与能量密度$\epsilon$的关系为
\begin{equation}\label{eq3}
  P(\mu) = \frac{1}{3} \epsilon(\mu) = \frac{1}{3} \left[ \epsilon_q(\mu) + \epsilon_{\bar{q}} (\mu) \right].
\end{equation}

我们将\eqref{eq2}求化学势$\mu(\bm{r})$和势能$V(\bm{r})$的关系. 为此, 把等离子体中夸克和反夸克的数密度以及能量密度用化学势$\mu$和温度$T$具体写出来. 由于平衡态的密度由$\mu = 0$描述, 我们把密度展开到$\mu$的一次项, 有
\begin{equation}\label{eq4}
  \begin{split}
    n_q(\mu) &= \frac{g_q}{2\pi^2} \int_0^\infty \frac{p_0^2 \diff p_0}{1 + e^{(p_0 - \mu)/T}} \\
    &= \frac{g_q T^3}{2\pi^2} \int_0^\infty \frac{z^2 \diff z}{1 + e^{z-(\mu/T)}} \\
    &= \frac{g_q T^3}{2\pi^2} \int_0^\infty z^2 \diff z \left[ \frac{1}{1 + e^z} - \frac{\mu}{T} \frac{\diff}{\diff z} \frac{1}{1 + e^z} \right] + \cdots \\
    &= \frac{g_q T^3}{2\pi^2} \int_0^\infty \diff z \left[ \frac{z^2}{1+e^z} + \frac{\mu}{T} \frac{2z}{1 + e^z}  \right] + \cdots
  \end{split}
\end{equation}
可以证明
\begin{equation}
  \int_0^\infty \diff z \frac{z^{x-1}}{1+e^z} = (1-2^{1-x}) \varGamma(x) \zeta(x),
\end{equation}
其中$\zeta(x)$是黎曼$\zeta$函数.

从这些结果, 我们得到在等离子体中夸克的数密度为
\begin{equation}\label{eq5}
  n_q(\mu) = \frac{g_qT^3}{2\pi^2} \left[ \frac{3}{2} \zeta(3) + \frac{\mu}{T} \frac{\pi^2}{6} \right].
\end{equation}
我们可以用同样的方式得到夸克的能量密度, 有
\begin{equation}
  \begin{split}
    \epsilon_q(\mu) &= \frac{g_q}{2\pi^2} \int_0^\infty \frac{p_0^3 \diff p_0}{1+e^{(p_0 - \mu)/T}} = \frac{g_qT^4}{2\pi^2} \int_0^\infty \frac{z^3\diff z}{1+e^{z-(\mu/T)}} \\
    &= \frac{g_qT^4}{2\pi^2} \int_0^\infty z^3 \diff z \left[ \frac{1}{1+e^z} - \frac{\mu}{T} \frac{\diff }{\diff z} \frac{1}{1+e^z} + \frac{1}{2}\left(\frac{\mu}{T}\right)^2 \frac{\diff^2}{\diff z^2} \frac{1}{1 + e^z} \right] + \cdots \\
    &= \frac{g_qT^4}{2\pi^2} \int_0^\infty \diff z \left[ \frac{z^3}{1+e^z} + \frac{\mu}{T} \frac{3z^2}{1+e^z} + \frac{1}{2}\left(\frac{\mu}{T}\right)^2 \frac{6z}{1+e^z} \right] + \cdots
  \end{split}
\end{equation}
展开到化学势的第二项, 在等离子体中夸克的能量密度为
\begin{equation}\label{eq6}
  \epsilon_q(\mu) = \frac{g_q T^4}{2\pi^2} \left[ \frac{7}{4} \frac{\pi^4}{30} + \frac{\mu}{T} \frac{9}{2} \zeta(3) + \frac{1}{2} \left( \frac{\mu}{T} \right)^2 \frac{\pi^2}{2} \right].
\end{equation}
在给定化学势的情况下, 我们能够得到反夸克的能量密度. 由于存在反夸克对应于在负能态缺少夸克, 所以反夸克的数密度为
\begin{equation}
  \begin{split}
    n_{\bar{q}}(\mu) &= \frac{g_q}{2\pi^2} \int_{-\infty}^0 p_0^2 \diff p_0 \left[ 1 - \frac{1}{1 + e^{(p_0-\mu)/T}} \right] \\
    &= \frac{g_q}{2\pi^2} \int_{-\infty}^0 p_0^2 \diff p_0 \frac{e^{(p_0-\mu)/T}}{1+e^{(p_0-\mu)/T}} \\
    &= \frac{g_q}{2\pi^2} \int_{-\infty}^0 p_0^2 \diff p_0 \frac{1}{1 + e^{-(p_0-\mu)/T}}.
  \end{split}
\end{equation}
做变换$p_0 = -\bar{p}_0, \bar{p}_0 \geq 0$, 我们有
\begin{equation}\label{eq7}
  n_{\bar{q}}(\mu) = \frac{g_q}{2\pi^2} \int_0^\infty \bar{p}_0^2 \diff \bar{p}_0 \frac{1}{1 + e^{(\bar{p}_0+\mu)/T}}.
\end{equation}
比较\eqref{eq7}式和\eqref{eq4},\eqref{eq5}式, 我们得到反夸克的数密度为
\begin{equation} \label{eq8}
  n_{\bar{q}}(\mu) = \frac{g_q T^3}{2\pi^2} \left[ \frac{3}{2} \zeta(3) - \frac{\mu}{T} \frac{\pi^2}{6} \right].
\end{equation}
同样的反夸克的能量密度为
\begin{equation}\label{eq9}
  \begin{split}
    \epsilon_{\bar{q}}(\mu) &= \frac{g_q}{2\pi^2} \int_0^\infty \frac{\bar{p}_0^3 \diff \bar{p}_0}{1 + e^{(\bar{p}_0 + \mu)/T}} \\
    &= \frac{g_q T^4}{2\pi^2} \left[ \frac{7}{4} \frac{\pi^4}{30} - \frac{\mu}{T} \frac{9}{2}\zeta(3) + \frac{1}{2} \left( \frac{\mu}{T} \right)^2 \frac{\pi^2}{2}  \right].
  \end{split}
\end{equation}

从所有这些是化学势显函数的量, 我们可以得到关于$\mu(r)$的方程. 将\eqref{eq3}式在$\mu = 0$展开, 并注意到$\mu$的线性项为零, 我们有
\begin{equation}
  P(\mu) = \frac{1}{3} \left[ \epsilon(\mu = 0) + \frac{1}{2}\mu^2 \frac{\partial^2 \epsilon}{\partial \mu^2} \right],
\end{equation}
其中$\partial^2\epsilon/\partial \mu^2$是在$\mu = 0$求值, 根据\eqref{eq6}和\eqref{eq9}式,
\begin{equation}
  \frac{\partial^2 \epsilon}{\partial \mu^2} = \frac{g_qT^4}{2\pi^2} \frac{\pi^2}{T^2}.
\end{equation}
因此, \eqref{eq2}式变为
\begin{equation}\label{eq11}
  \nabla \frac{1}{6} \mu^2 \frac{\partial^2\epsilon}{\partial \mu^2} - q\left[ n_q(\mu) - n_{\bar{q}}(\mu) \right] \nabla V = 0.
\end{equation}
其中$n_q(\mu) - n_{\bar{q}}(\mu)$是化学势$\mu$的函数, 从\eqref{eq5}和\eqref{eq8}式, 我们有
\begin{equation}\label{eq12}
  n_q(\mu)-n_{\bar{q}}(\mu) = n_q(\mu = 0) + \mu \frac{\partial n_q}{\partial \mu} - n_{\bar{q}}(\mu = 0) - \mu \frac{\partial n_{\bar{q}}}{\partial \mu} = 2\mu \frac{\partial n_q}{\partial \mu},
\end{equation}
其中的$\partial n_q / \partial \mu$是在$\mu = 0$求值, 有\eqref{eq5}式给出
\begin{equation}
  \frac{\partial n_q}{\partial \mu} = \frac{g_q T^3}{2\pi^2} \frac{\pi^2}{6T}.
\end{equation}
在方程\eqref{eq12}中, 我们用到了$n_q = n_{\bar{q}}$, 以及在$\mu = 0, \partial n_{\bar{q}} / \partial \mu = - \partial n_q / \partial \mu$. 平衡条件\eqref{eq11}变为
\begin{equation}
  \frac{1}{3} \frac{\partial^2 \epsilon}{\partial \mu^2} \mu \nabla \mu - 2q\mu \frac{\partial n_q}{\partial \mu} \nabla V = 0.
\end{equation}
它在下式成立的情况下被满足,
\begin{equation}\label{eq14}
  \mu \frac{\partial^2\epsilon}{\partial \mu^2} - 6q \frac{\partial n_q}{\partial \mu} V = (\text{与}r\text{有关的常数}).
\end{equation}
当$r \rightarrow \infty$时, $\mu$和$V$趋于零, \eqref{eq14}式中的常数为零.

\eqref{eq14}给出的是化学势$\mu$和势能$V$之间的关系. 化学势与夸克和反夸克的数密度有关, 而数密度又通过泊松方程与势能$V_q$和$V_{\bar{q}}$有关. 因此, 我们可以把\eqref{eq14}重写为只是势能$V$的函数. 由\eqref{eq12}式, 我们可以将$\mu$表示为密度的函数
\begin{equation}
  \mu = \frac{n_q(\mu) - n_{\bar{q}}(\mu)}{2 \frac{\partial n_q}{\partial \mu}},
\end{equation}
根据泊松方程, 夸克的密度$n_q(\mu)$和这些夸克产生的势能的关系为
\begin{equation}
  \nabla^2 V_q = -q n_q (\mu).
\end{equation}
反夸克的密度$n_{\bar{q}}(\mu)$和这些反夸克产生的势能关系为
\begin{equation}
  \nabla^2 V_{\bar{q}} = -(-q)n_{\bar{q}} (\mu).
\end{equation}
因此, 化学势与$V_q$和$V_{\bar{q}}$的关系为
\begin{equation}
  \mu = \frac{-\nabla^2 V_q - \nabla^2 V_{\bar{q}}}{2q \frac{\partial n_q}{\partial \mu}}
\end{equation}
将上式带入\eqref{eq14}式, 我们得到
\begin{equation}
  \nabla^2 V_q + \nabla^2 V_{\bar{q}} + 12 q^2 \left( \frac{\partial n_q}{\partial \mu} \right)^2 \left( \frac{\partial^2\epsilon}{\partial \mu^2} \right)^{-1} V = 0,
\end{equation}
或
\begin{equation}\label{eq15}
  \nabla^2 (V_q + V_{\bar{q}}) + m_D^2 V = 0,
\end{equation}
其中$m_D$是德拜屏蔽质量, 定义为
\begin{equation}
  \begin{split}
    m_D^2 &= 12 q^2 \left( \frac{\partial n_q}{\partial \mu} \right)^2 \left( \frac{\partial^2 \epsilon}{\partial \mu^2} \right)^{-1} = \frac{g_q q^2 T^2}{6} \\
    &= \frac{\pi^2}{9\times 1.202} \frac{q^2 (n_q + n_{\bar{q}})}{T}.
  \end{split}
\end{equation}
利用\eqref{eq1}式, 我们可以将\eqref{eq15}式重新写为
\begin{equation}
  \nabla^2(V - V_0) + m_D^2 V = 0.
\end{equation}
势能$V_0(\bm{r})$满足点源的泊松方程,
\begin{equation}
  \nabla^2 V + m_D^2 V = -q \delta(\bm{r}).
\end{equation}
由此给出Yukawa势
\begin{equation}
  V(r) = \frac{q}{4\pi} \frac{e^{-m_Dr}}{r} = \frac{q}{4\pi} \frac{e^{-r/\lambda_D}}{r},
\end{equation}
其中德拜屏蔽长度$\lambda_D$是德拜屏蔽质量的倒数,
\begin{equation}
  \lambda_D^2 = 1 / m_D^2 = \frac{6}{g_q} \frac{1}{q^2T^2} = \frac{9 \times 1.202 T}{\pi^2 q^2 (n_q + n_{\bar{q}})}.
\end{equation}


\end{answer}
\end{document}